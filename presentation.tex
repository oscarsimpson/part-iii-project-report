\documentclass{beamer}
\usepackage{latexsym}
\usepackage{amssymb}
\usepackage{physics}
\usepackage{siunitx}
\usepackage{slashed}
\usepackage[plain]{fullpage}
\usepackage{titlesec}
\usepackage{marvosym}
\usepackage[usenames,dvipsnames]{color}
\usepackage{verbatim}
\usepackage{enumitem}
\usepackage[hidelinks]{hyperref}
\usepackage{fancyhdr}
\usepackage[english]{babel}

\usepackage[font=small, labelfont=bf]{caption}

\usepackage{tabularx}
\usepackage{booktabs}
\usepackage{graphicx}
\graphicspath{{./images/}}

\usepackage{csquotes}
\usepackage[sorting=none]{biblatex}
\addbibresource{refs.bib}

\newcommand{\code}{\texttt}

\newcommand{\D}[1]{\mathinner{\operatorname{\mathrm{\mathcal{D}}}\!#1}}

\newlength{\imgwidth}
\setlength{\imgwidth}{0.8\textwidth}

\graphicspath{\
{../PartIIIProject/preliminary/images/}
{../PartIIIProject/fitter/images_separate/MeV/2pt_hisq_coarse_D_Gold_K_p0_1053conf/}
{../PartIIIProject/fitter/images_separate/MeV/2pt_hisq_msml5_fine_K_zeromom_D_Gold_nongold_495conf/}
{../PartIIIProject/fitter/images_separate/MeV/2pt_hisq_very-coarse_D_Gold_K_1000conf/}
{./images/}}

\usetheme{Madrid}



\title[]{Obtaining $K$ and $D$ meson properties from lattice QCD}
\author[Oscar Simpson]{Oscar Simpson \\ Supervisor: Dr.\ B.\ Chakraborty}
\date{May 2021}

\begin{document}

\frame{\titlepage}


\begin{frame}
\frametitle{QCD Recap}
We begin with a recap of QCD\cite{2010QFT, 2015Colquhoun}.
\begin{itemize}
    \item[]<1-> 
    \begin{equation*}
    \mathcal{L} = \sum_f \psi^\dagger_f (i\slashed{D}-m_f) \psi_f - \frac{1}{4}F^a_{\mu\nu} F^{\mu\nu}_a
    \end{equation*}
    
    \item[]<2->
    \begin{equation*}
    F^a_{\mu\nu} = \partial_\mu A_\nu^a - \partial_\nu A_\mu^a - g_s f_{abc} A_\mu^b A_\nu^c
    \end{equation*}
\end{itemize}
\end{frame}

\begin{frame}
\frametitle{Lattice QCD Recap}
Now we move on to a recap of lattice QCD\cite{1948Feynman, 2015Colquhoun}.
\begin{itemize}
    \item[]<1->
    \begin{alignat*}{3}
    \expval{\mathcal{O}} &= \dfrac{1}{\mathcal{Z}}&&\int\D{\psi} \mathcal{O}  &&\exp(i \int \dd[4]{x} \mathcal{L}(\psi, \partial \psi)) \\
    \mathcal{Z}          &=                       &&\int\D{\psi}              &&\exp(i \int \dd[4]{x} \mathcal{L}(\psi, \partial \psi))
    \end{alignat*}

\end{itemize}
    
\end{frame}

\begin{frame}
\frametitle{Lattice QCD Recap}
\begin{figure}
    \centering
    \includegraphics[width=\imgwidth]{lattice.png}
    \caption{An example of a discretisation. Red crosses indicate quark fields, and blue dots indicate gauge fields.}
\end{figure}
\end{frame}

\begin{frame}
    \frametitle{Highly Improved Staggered Quark}
    \begin{itemize}
        \item Fermions placed on a finite lattice give "doubled" fermions\cite{2016Chakraborty}.
        \item[]
            \begin{equation}
    M^{-1} = \frac{-i\gamma_\mu \sin(ap_\mu)/a+m}{\sum_{\mu=0}^3 \sin^2(ap_\mu)/a^2 + m^2}
    \end{equation}
    \item These are the reflections of the fermion mass within the Brillouin zone in each of the 4 spacetime dimensions, which gives $16=2^4$ extra modes, called "tastes".
    \item To remove this problem, Wilson\cite{1974Wilson} used a construction to give these tastes infinite mass, such that they decouple in the continuum limit. 
    \item A procedure called "staggering" transformation\cite{mason2006high} allows us to bring the system down to only 4 doublers.
    \item With some further optimizations this gives the Highly Improved Staggered Quark (HISQ) action, which has very small discretisation errors.
    \end{itemize}
    \end{frame}

\begin{frame}
\frametitle{Data}
\begin{itemize}
    \item<1-> Data was provided by Dr.\ B.\ Chakraborty.
    \item<1-> Data has been produced using isotropic lattice NRQCD\cite{chakraborty2021improved}.
    \item[]<2-> 
        \begin{table}
            \centering
            \begin{tabular}{l S[table-format=1.5] S[table-format=2.0] S[table-format=1.3]}
Label       & {Spacing $a$ (\si{fm})}   & {Lattice size (time)} & {$m_l / m_s$}\\
\midrule 
Very Coarse & 0.1509                    & 48                    & 0.036\\
Coarse      & 0.12404                   & 64                    & 0.2\\
Fine        & 0.09023                   & 96                    & 0.2\\
            \end{tabular}
            \caption{Parameters of the data used in the main part of this project. Corresponding to sets 1, 5, 6 from table I in~\cite{chakraborty2021improved} respectively.}
        \end{table}
    \item<3-> Finite lattice leads to discretisation errors: must take continuum limit.
    \item<4-> Computations requirements mean we must use unphysical light quark masses: must take chiral limit.
\end{itemize}
\end{frame}

\begin{frame}
\frametitle{Statistical Analysis: Jacknife averaging}
\begin{itemize}
    \item<1-> Given a sample of $n$ values $x_i \sim F$, we want to find a way to construct best estimators for $F$.
    \item<2-> For example, estimates for the mean:
        \begin{itemize}
            \item Construct $\overline{x}_i = \sum_{j\neq i} \frac{x_j}{n-1}$.
            \item $\overline{x} = \sum_i\frac{\overline{x}_i}{n}$ is in fact an unbiased estimator for the mean of $F$.
        \end{itemize}
    \item<3-> A more complex variant of this procedure can be used to find an unbiased estimator for any parameter of $F$\cite{efron1982jackknife}.
\end{itemize}
\end{frame}

\begin{frame}
    \frametitle{Statistical Analysis: Bayesian analysis}
    \begin{itemize}
        \item<1-> We use the \code{corrfitter}\cite{lepage2020corrfitter} Python package.
        \item<2-> Perform least-squares regression on 2-point correlator data.
        \item[]<2-> 
            \begin{equation*}
                G(t) = \sum_n a_n^2 (e^{-E_n t} + e^{-E_n(T-t)}) + {(-1)}^t \sum_n {a_o}_n^2 (e^{-{E_o}_n t} + e^{{-E_o}_n (T-t)})
            \end{equation*}
\end{itemize}
\end{frame}

\begin{frame}
    \frametitle{Preliminary Study: effective mass}
    \begin{equation*}
        G(t) = \sum_n a_n^2 (e^{-E_n t} + e^{-E_n(T-t)}) + {(-1)}^t \sum_n {a_o}_n^2 (e^{-{E_o}_n t} + e^{{-E_o}_n (T-t)})
    \end{equation*}
    Assuming only the first term dominates, and there is little contribution from the oscillatory terms, then we have:
    \begin{equation*}
        m_{\mathrm{eff}}(t) = \log(\frac{G(t)}{G(t+1)}).
    \end{equation*}
\end{frame}

\begin{frame}
    \frametitle{Preliminary Study: effective mass}
    \begin{equation*}
        m_{\mathrm{eff}}(t) = \log(\frac{G(t)}{G(t+1)})
    \end{equation*}
    \begin{figure}
        \centering
        \includegraphics[width=1.3\imgwidth]{correlation_2pt_hisq_msml5_fine_D_nongold_489conf.png}
        \caption{Jacknife averaged correlation data for $D^0$ meson on a fine lattice.}
    \end{figure}
\end{frame}

\begin{frame}
    \frametitle{Preliminary Study: effective mass}
    \begin{equation*}
        m_{\mathrm{eff}}(t) = \log(\frac{G(t)}{G(t+1)})
    \end{equation*}
    \begin{figure}
        \centering
        \includegraphics[width=1.3\imgwidth]{meff_2pt_hisq_msml5_fine_D_nongold_489conf.png}
        \caption{Estimate for effective mass of $D^0$ meson.}
    \end{figure}
\end{frame}

\begin{frame}
    \frametitle{Data Analysis: fine lattice}
    \begin{itemize}
        \item Fine dataset has $a \approx \SI{0.09}{fm}$ and $m_l \approx 0.2 m_s$.
        \item K meson clearly converges for $n \geq 4$.
    \end{itemize}
    \begin{figure}
        \centering
        \includegraphics[width=1.3\imgwidth]{fine_tmin3_K.png}
        \caption{Results of fitting K meson dataset on a fine lattice.}
    \end{figure}
\end{frame}

\begin{frame}
    \frametitle{Data Analysis: fine lattice}
    Combined fit quality plot shows that the fitting converged for all $t_\mathrm{min}$ and $n \geq 4$.
    \begin{figure}
        \centering
        \includegraphics[width=1.3\imgwidth]{fine_fitQuality.png}
        \caption{Fitting quality for fine lattice.}
    \end{figure}
\end{frame}

\begin{frame}
    \frametitle{Data Analysis: coarse lattice}
    \begin{itemize}
        \item Coarse dataset has $a \approx \SI{0.12}{fm}$ and $m_l \approx 0.2 m_s$.
        \item K meson converges for $n \geq 3$.
    \end{itemize}
    \begin{figure}
        \centering
        \includegraphics[width=1.3\imgwidth]{coarse_tmin3_K.png}
        \caption{Results of fitting K meson dataset on a coarse lattice.}
    \end{figure}
\end{frame}

\begin{frame}
    \frametitle{Data Analysis: coarse lattice}
    Combined fit quality plot shows that the fitting converged for all $t_\mathrm{min}$ and varying $n$.
    \begin{figure}
        \centering
        \includegraphics[width=1.3\imgwidth]{coarse_fitQuality.png}
        \caption{Fitting quality for coarse lattice.}
    \end{figure}
\end{frame}

\begin{frame}
    \frametitle{Data Analysis: very coarse lattice}
    \begin{itemize}
        \item Fine dataset has $a \approx \SI{0.15}{fm}$ and $m_l \approx 0.036 m_s$.
        \item K meson appears to converge for $n \geq 2$.
    \end{itemize}
    \begin{figure}
        \centering
        \includegraphics[width=1.3\imgwidth]{very-coarse_tmin3_K.png}
        \caption{Results of fitting K meson dataset on a very coarse lattice.}
    \end{figure}
\end{frame}

\begin{frame}
    \frametitle{Data Analysis: very coarse lattice}
    Combined fit quality plot shows that the fitting converged for a very limited selection of $t_\mathrm{min}$ and $n$.
    \begin{figure}
        \centering
        \includegraphics[width=1.3\imgwidth]{very-coarse_fitQuality.png}
        \caption{Fitting quality for very coarse lattice.}
    \end{figure}
\end{frame}

\begin{frame}
    \frametitle{Results}
    For consistency between plots, $t_\mathrm{min}=6$ for $D$ mesons, and $n=4$ for all results, have been selected.
    \begin{table}
    \centering
    \tiny
    \begin{tabular}{l S[table-format=1.2] S[table-format=1.3] S[table-format=3.2(2)] S[table-format=4.1(2)] S[table-format=1.5(2)] S[table-format=1.4(2)]}
        \toprule
                    &                   &                   & \multicolumn{2}{c}{$E_0$ (MeV)}       & \multicolumn{2}{c}{$|a_0|$}\\
        \cmidrule(lr){4-5}\cmidrule(lr){6-7}
        Label       & {$a$ (\si{fm})}   & {$m_l / m_s$}     & {K}               & {D}               & {K}                & {D}\\
        \midrule
        Very Coarse & 0.15              & 0.036             & 495.92(08)        & 1887.3(51)        & 0.28101(11)        & 0.2286(65)\\
        Coarse      & 0.12              & 0.2               & 528.28(15)        & 1893.0(15)        & 0.60853(37)        & 0.1880(12)\\
        Fine        & 0.09              & 0.2               & 530.16(24)        & 1889.3(11)        & 0.13978(12)        & 0.1242(07)\\
        \bottomrule
    \end{tabular}
    \normalsize
    \caption{Results from fitting all datasets. Uncertainties given in parentheses are statistical.}
    \end{table}
    These are reasonable mass values as literature\cite{zyla2020review} quotes $m_K \approx \SI{498}{MeV}$ and $m_D \approx \SI{1865}{MeV}$.
\end{frame}

\begin{frame}
    \frametitle{Chiral/Continuum Extrapolation}
    \begin{itemize}
        \item Full extrapolation\cite{chakraborty2017nonperturbative} requires 7 pairs of parameters.
        \item As we have very few datasets, we restrict to first order in each parameter. 
        \item[]
        \begin{equation*}
            m = m_\mathrm{phys} \Big(1 + c_\delta \frac{m_l}{m_s}\Big)\Big(1 + c_{a^2} a^2\Big)
        \end{equation*}
    \end{itemize}
\end{frame}

\begin{frame}
    \frametitle{Chiral/Continuum Extrapolation}
    The fit has $\chi^2 / \mathrm{dof} = 0.23$ so we have a good quality fit, and importantly not an overfit.
    \begin{table}
    \centering
    \tiny
    \begin{tabular}{l S[table-format=1.2] S[table-format=1.3] S[table-format=3.2(2)] S[table-format=4.1(2)] S[table-format=1.5(2)] S[table-format=1.4(2)]}
        \toprule
                    &                   &                   & \multicolumn{2}{c}{$E_0$ (MeV)}       & \multicolumn{2}{c}{$|a_0|$}\\
        \cmidrule(lr){4-5}\cmidrule(lr){6-7}
        Label       & {$a$ (\si{fm})}   & {$m_l / m_s$}     & {K}               & {D}               & {K}                & {D}\\
        \midrule
        Very Coarse & 0.15              & 0.036             & 495.92(08)        & 1887.3(51)        & 0.28101(11)        & 0.2286(65)\\
        Coarse      & 0.12              & 0.2               & 528.28(15)        & 1893.0(15)        & 0.60853(37)        & 0.1880(12)\\
        Fine        & 0.09              & 0.2               & 530.16(24)        & 1889.3(11)        & 0.13978(12)        & 0.1242(07)\\
        \bottomrule
    \end{tabular}
    \normalsize
    \caption{Comparison of observed and predicted mass values, including the extrapolated chiral/continuum limit values.\label{table:extrapolated_mass}}
    \end{table}
    The extrapolated values are consistent with results from the literature\cite{zyla2020review}, $m_{K^0} = \SI{497.611(13)}{MeV}$ and $m_{D^0} = \SI{1864.83(5)}{MeV}$ to within $3\sigma$ of our statistical uncertainty.
\end{frame}

\begin{frame}
    \frametitle{Remaining Work}
    \begin{itemize}
        \item Compute partial decay widths, using amplitudes $a_0$.
    \end{itemize}
\end{frame}

\begin{frame}
\frametitle{Bibliography}
\renewcommand*{\bibfont}{\tiny}
\printbibliography[heading=none]
\end{frame}

\end{document}
